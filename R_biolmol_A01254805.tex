% Options for packages loaded elsewhere
\PassOptionsToPackage{unicode}{hyperref}
\PassOptionsToPackage{hyphens}{url}
%
\documentclass[
]{article}
\usepackage{amsmath,amssymb}
\usepackage{lmodern}
\usepackage{iftex}
\ifPDFTeX
  \usepackage[T1]{fontenc}
  \usepackage[utf8]{inputenc}
  \usepackage{textcomp} % provide euro and other symbols
\else % if luatex or xetex
  \usepackage{unicode-math}
  \defaultfontfeatures{Scale=MatchLowercase}
  \defaultfontfeatures[\rmfamily]{Ligatures=TeX,Scale=1}
\fi
% Use upquote if available, for straight quotes in verbatim environments
\IfFileExists{upquote.sty}{\usepackage{upquote}}{}
\IfFileExists{microtype.sty}{% use microtype if available
  \usepackage[]{microtype}
  \UseMicrotypeSet[protrusion]{basicmath} % disable protrusion for tt fonts
}{}
\makeatletter
\@ifundefined{KOMAClassName}{% if non-KOMA class
  \IfFileExists{parskip.sty}{%
    \usepackage{parskip}
  }{% else
    \setlength{\parindent}{0pt}
    \setlength{\parskip}{6pt plus 2pt minus 1pt}}
}{% if KOMA class
  \KOMAoptions{parskip=half}}
\makeatother
\usepackage{xcolor}
\usepackage[margin=1in]{geometry}
\usepackage{color}
\usepackage{fancyvrb}
\newcommand{\VerbBar}{|}
\newcommand{\VERB}{\Verb[commandchars=\\\{\}]}
\DefineVerbatimEnvironment{Highlighting}{Verbatim}{commandchars=\\\{\}}
% Add ',fontsize=\small' for more characters per line
\usepackage{framed}
\definecolor{shadecolor}{RGB}{248,248,248}
\newenvironment{Shaded}{\begin{snugshade}}{\end{snugshade}}
\newcommand{\AlertTok}[1]{\textcolor[rgb]{0.94,0.16,0.16}{#1}}
\newcommand{\AnnotationTok}[1]{\textcolor[rgb]{0.56,0.35,0.01}{\textbf{\textit{#1}}}}
\newcommand{\AttributeTok}[1]{\textcolor[rgb]{0.77,0.63,0.00}{#1}}
\newcommand{\BaseNTok}[1]{\textcolor[rgb]{0.00,0.00,0.81}{#1}}
\newcommand{\BuiltInTok}[1]{#1}
\newcommand{\CharTok}[1]{\textcolor[rgb]{0.31,0.60,0.02}{#1}}
\newcommand{\CommentTok}[1]{\textcolor[rgb]{0.56,0.35,0.01}{\textit{#1}}}
\newcommand{\CommentVarTok}[1]{\textcolor[rgb]{0.56,0.35,0.01}{\textbf{\textit{#1}}}}
\newcommand{\ConstantTok}[1]{\textcolor[rgb]{0.00,0.00,0.00}{#1}}
\newcommand{\ControlFlowTok}[1]{\textcolor[rgb]{0.13,0.29,0.53}{\textbf{#1}}}
\newcommand{\DataTypeTok}[1]{\textcolor[rgb]{0.13,0.29,0.53}{#1}}
\newcommand{\DecValTok}[1]{\textcolor[rgb]{0.00,0.00,0.81}{#1}}
\newcommand{\DocumentationTok}[1]{\textcolor[rgb]{0.56,0.35,0.01}{\textbf{\textit{#1}}}}
\newcommand{\ErrorTok}[1]{\textcolor[rgb]{0.64,0.00,0.00}{\textbf{#1}}}
\newcommand{\ExtensionTok}[1]{#1}
\newcommand{\FloatTok}[1]{\textcolor[rgb]{0.00,0.00,0.81}{#1}}
\newcommand{\FunctionTok}[1]{\textcolor[rgb]{0.00,0.00,0.00}{#1}}
\newcommand{\ImportTok}[1]{#1}
\newcommand{\InformationTok}[1]{\textcolor[rgb]{0.56,0.35,0.01}{\textbf{\textit{#1}}}}
\newcommand{\KeywordTok}[1]{\textcolor[rgb]{0.13,0.29,0.53}{\textbf{#1}}}
\newcommand{\NormalTok}[1]{#1}
\newcommand{\OperatorTok}[1]{\textcolor[rgb]{0.81,0.36,0.00}{\textbf{#1}}}
\newcommand{\OtherTok}[1]{\textcolor[rgb]{0.56,0.35,0.01}{#1}}
\newcommand{\PreprocessorTok}[1]{\textcolor[rgb]{0.56,0.35,0.01}{\textit{#1}}}
\newcommand{\RegionMarkerTok}[1]{#1}
\newcommand{\SpecialCharTok}[1]{\textcolor[rgb]{0.00,0.00,0.00}{#1}}
\newcommand{\SpecialStringTok}[1]{\textcolor[rgb]{0.31,0.60,0.02}{#1}}
\newcommand{\StringTok}[1]{\textcolor[rgb]{0.31,0.60,0.02}{#1}}
\newcommand{\VariableTok}[1]{\textcolor[rgb]{0.00,0.00,0.00}{#1}}
\newcommand{\VerbatimStringTok}[1]{\textcolor[rgb]{0.31,0.60,0.02}{#1}}
\newcommand{\WarningTok}[1]{\textcolor[rgb]{0.56,0.35,0.01}{\textbf{\textit{#1}}}}
\usepackage{graphicx}
\makeatletter
\def\maxwidth{\ifdim\Gin@nat@width>\linewidth\linewidth\else\Gin@nat@width\fi}
\def\maxheight{\ifdim\Gin@nat@height>\textheight\textheight\else\Gin@nat@height\fi}
\makeatother
% Scale images if necessary, so that they will not overflow the page
% margins by default, and it is still possible to overwrite the defaults
% using explicit options in \includegraphics[width, height, ...]{}
\setkeys{Gin}{width=\maxwidth,height=\maxheight,keepaspectratio}
% Set default figure placement to htbp
\makeatletter
\def\fps@figure{htbp}
\makeatother
\setlength{\emergencystretch}{3em} % prevent overfull lines
\providecommand{\tightlist}{%
  \setlength{\itemsep}{0pt}\setlength{\parskip}{0pt}}
\setcounter{secnumdepth}{-\maxdimen} % remove section numbering
\ifLuaTeX
  \usepackage{selnolig}  % disable illegal ligatures
\fi
\IfFileExists{bookmark.sty}{\usepackage{bookmark}}{\usepackage{hyperref}}
\IfFileExists{xurl.sty}{\usepackage{xurl}}{} % add URL line breaks if available
\urlstyle{same} % disable monospaced font for URLs
\hypersetup{
  pdftitle={R\_biolmol\_A01254805},
  pdfauthor={Daniel Barreras, Yair Beltrán, Victor Symonds},
  hidelinks,
  pdfcreator={LaTeX via pandoc}}

\title{R\_biolmol\_A01254805}
\author{Daniel Barreras, Yair Beltrán, Victor Symonds}
\date{2023-04-14}

\begin{document}
\maketitle

\hypertarget{parte-1}{%
\subsection{PARTE 1}\label{parte-1}}

\begin{quote}
Crea una función que genere una secuencia aleatoria de nucleótidos de
ADN (A, T, G y C) de tamaño ``n''. Ejecútala solicitando una secuencia
aleatoria de ADN de 30 nucleótidos y muestra el resultado impreso en
consola.
\end{quote}

\begin{Shaded}
\begin{Highlighting}[]
\NormalTok{crearSecuenciaADN }\OtherTok{\textless{}{-}} \ControlFlowTok{function}\NormalTok{(}\AttributeTok{n =} \DecValTok{30}\NormalTok{) \{}
  
\NormalTok{  nucleotidos }\OtherTok{\textless{}{-}} \FunctionTok{c}\NormalTok{(}\StringTok{"A"}\NormalTok{, }\StringTok{"T"}\NormalTok{, }\StringTok{"G"}\NormalTok{, }\StringTok{"C"}\NormalTok{)}
\NormalTok{  secuencia }\OtherTok{\textless{}{-}} \FunctionTok{sample}\NormalTok{(nucleotidos, n, }\AttributeTok{replace=}\ConstantTok{TRUE}\NormalTok{)}
  
  \FunctionTok{return}\NormalTok{(}\FunctionTok{paste}\NormalTok{(secuencia, }\AttributeTok{collapse=}\StringTok{""}\NormalTok{))}
\NormalTok{\}}


\NormalTok{secuenciaAleatoria }\OtherTok{\textless{}{-}} \FunctionTok{crearSecuenciaADN}\NormalTok{(}\DecValTok{30}\NormalTok{);}
\FunctionTok{print}\NormalTok{(secuenciaAleatoria)}
\end{Highlighting}
\end{Shaded}

\begin{verbatim}
## [1] "CAGGTACTGGTAGGTGAGCAAGCATAAACC"
\end{verbatim}

\begin{quote}
Crea una función que calcule el tamaño de una secuencia de ADN.
Utilízala para calcular el tamaño de la secuencia que generaste en el
punto 1 y muestra el resultado impreso en consola.
\end{quote}

\begin{Shaded}
\begin{Highlighting}[]
\NormalTok{calcularTamañoSecuencia }\OtherTok{\textless{}{-}} \ControlFlowTok{function}\NormalTok{(secuencia) \{}
  \FunctionTok{return}\NormalTok{(}\FunctionTok{nchar}\NormalTok{(secuencia))}
\NormalTok{\}}

\NormalTok{tamañoSecuencia }\OtherTok{\textless{}{-}}\NormalTok{ calcularTamañoSecuencia(secuenciaAleatoria)}

\FunctionTok{cat}\NormalTok{(}\StringTok{"El tamaño de la secuencia de ADN es:"}\NormalTok{, tamañoSecuencia)}
\end{Highlighting}
\end{Shaded}

\begin{verbatim}
## El tamaño de la secuencia de ADN es: 30
\end{verbatim}

\begin{quote}
Crea una función que recibe una secuencia de DNA e imprime el porcentaje
de cada base (A, C, G y T) en la secuencia. Ejecútala sobre la secuencia
que generaste en el punto 1 y muestra el resultado impreso en consola.
\end{quote}

\begin{Shaded}
\begin{Highlighting}[]
\NormalTok{calcPorcentajeBases }\OtherTok{\textless{}{-}} \ControlFlowTok{function}\NormalTok{(secuencia)\{}
  
\NormalTok{  n }\OtherTok{\textless{}{-}} \FunctionTok{nchar}\NormalTok{(secuencia)}
\NormalTok{  a }\OtherTok{\textless{}{-}} \FunctionTok{sum}\NormalTok{(}\FunctionTok{str\_count}\NormalTok{(secuencia, }\StringTok{"A"}\NormalTok{)) }
\NormalTok{  c }\OtherTok{\textless{}{-}} \FunctionTok{sum}\NormalTok{(}\FunctionTok{str\_count}\NormalTok{(secuencia, }\StringTok{"C"}\NormalTok{)) }
\NormalTok{  g }\OtherTok{\textless{}{-}} \FunctionTok{sum}\NormalTok{(}\FunctionTok{str\_count}\NormalTok{(secuencia, }\StringTok{"G"}\NormalTok{))}
\NormalTok{  t }\OtherTok{\textless{}{-}} \FunctionTok{sum}\NormalTok{(}\FunctionTok{str\_count}\NormalTok{(secuencia, }\StringTok{"T"}\NormalTok{))}

\NormalTok{  porA }\OtherTok{\textless{}{-}}\NormalTok{ a}\SpecialCharTok{/}\NormalTok{n }\SpecialCharTok{*} \DecValTok{100}
\NormalTok{  porC }\OtherTok{\textless{}{-}}\NormalTok{ c}\SpecialCharTok{/}\NormalTok{n }\SpecialCharTok{*} \DecValTok{100}
\NormalTok{  porG }\OtherTok{\textless{}{-}}\NormalTok{ g}\SpecialCharTok{/}\NormalTok{n }\SpecialCharTok{*} \DecValTok{100}
\NormalTok{  porT }\OtherTok{\textless{}{-}}\NormalTok{ t}\SpecialCharTok{/}\NormalTok{n }\SpecialCharTok{*} \DecValTok{100}
  
  \FunctionTok{cat}\NormalTok{(}\StringTok{"Porcentaje de bases A:"}\NormalTok{, }\FunctionTok{round}\NormalTok{(porA, }\AttributeTok{digits =} \DecValTok{2}\NormalTok{), }\StringTok{"\%}\SpecialCharTok{\textbackslash{}n}\StringTok{"}\NormalTok{)}
  \FunctionTok{cat}\NormalTok{(}\StringTok{"Porcentaje de bases C:"}\NormalTok{, }\FunctionTok{round}\NormalTok{(porC, }\AttributeTok{digits =} \DecValTok{2}\NormalTok{), }\StringTok{"\%}\SpecialCharTok{\textbackslash{}n}\StringTok{"}\NormalTok{)}
  \FunctionTok{cat}\NormalTok{(}\StringTok{"Porcentaje de bases G:"}\NormalTok{, }\FunctionTok{round}\NormalTok{(porG, }\AttributeTok{digits =} \DecValTok{2}\NormalTok{), }\StringTok{"\%}\SpecialCharTok{\textbackslash{}n}\StringTok{"}\NormalTok{)}
  \FunctionTok{cat}\NormalTok{(}\StringTok{"Porcentaje de bases T:"}\NormalTok{, }\FunctionTok{round}\NormalTok{(porT, }\AttributeTok{digits =} \DecValTok{2}\NormalTok{), }\StringTok{"\%}\SpecialCharTok{\textbackslash{}n}\StringTok{"}\NormalTok{)}
\NormalTok{\}}

\FunctionTok{calcPorcentajeBases}\NormalTok{(secuenciaAleatoria)}
\end{Highlighting}
\end{Shaded}

\begin{verbatim}
## Porcentaje de bases A: 33.33 %
## Porcentaje de bases C: 20 %
## Porcentaje de bases G: 30 %
## Porcentaje de bases T: 16.67 %
\end{verbatim}

\begin{quote}
Crea una función que recibe una hebra directa y regresa la hebra
inversa. Ejecútala sobre la secuencia que generaste en el punto 1 y
muestra el resultado impreso en consola.
\end{quote}

\begin{Shaded}
\begin{Highlighting}[]
\NormalTok{calcularHebraInversa }\OtherTok{\textless{}{-}} \ControlFlowTok{function}\NormalTok{(secuencia)\{}
  
  \FunctionTok{return}\NormalTok{(}\FunctionTok{stri\_reverse}\NormalTok{(secuencia))}
\NormalTok{\}}

\NormalTok{hebraInversa }\OtherTok{\textless{}{-}} \FunctionTok{calcularHebraInversa}\NormalTok{(secuenciaAleatoria)}
\FunctionTok{cat}\NormalTok{(}\StringTok{"Secuencia aleatoria: "}\NormalTok{, secuenciaAleatoria, }\StringTok{"}\SpecialCharTok{\textbackslash{}n}\StringTok{"}\NormalTok{)}
\end{Highlighting}
\end{Shaded}

\begin{verbatim}
## Secuencia aleatoria:  CAGGTACTGGTAGGTGAGCAAGCATAAACC
\end{verbatim}

\begin{Shaded}
\begin{Highlighting}[]
\FunctionTok{cat}\NormalTok{(}\StringTok{"Hebra inversa: "}\NormalTok{, hebraInversa)}
\end{Highlighting}
\end{Shaded}

\begin{verbatim}
## Hebra inversa:  CCAAATACGAACGAGTGGATGGTCATGGAC
\end{verbatim}

\begin{quote}
Crea una función qué recibe una hebra directa y obtiene la hebra
complementaria. Ejecútala sobre la secuencia que generaste en el punto 1
y muestra el resultado impreso en consola
\end{quote}

\begin{Shaded}
\begin{Highlighting}[]
\NormalTok{calculaHebraComplementaria }\OtherTok{\textless{}{-}} \ControlFlowTok{function}\NormalTok{(secuencia)\{}
  \FunctionTok{return}\NormalTok{(}\FunctionTok{chartr}\NormalTok{(}\StringTok{"ATCG"}\NormalTok{, }\StringTok{"TAGC"}\NormalTok{, secuencia))}
\NormalTok{\}}

\NormalTok{hebraComplementaria }\OtherTok{\textless{}{-}} \FunctionTok{calculaHebraComplementaria}\NormalTok{(secuenciaAleatoria)}

\FunctionTok{cat}\NormalTok{(}\StringTok{"Secuencia aleatoria: "}\NormalTok{, secuenciaAleatoria, }\StringTok{"}\SpecialCharTok{\textbackslash{}n}\StringTok{"}\NormalTok{)}
\end{Highlighting}
\end{Shaded}

\begin{verbatim}
## Secuencia aleatoria:  CAGGTACTGGTAGGTGAGCAAGCATAAACC
\end{verbatim}

\begin{Shaded}
\begin{Highlighting}[]
\FunctionTok{cat}\NormalTok{(}\StringTok{"Hebra complementaria:"}\NormalTok{, hebraComplementaria)}
\end{Highlighting}
\end{Shaded}

\begin{verbatim}
## Hebra complementaria: GTCCATGACCATCCACTCGTTCGTATTTGG
\end{verbatim}

\end{document}
